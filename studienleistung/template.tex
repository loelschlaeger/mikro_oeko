\documentclass{article}

\usepackage{hyperref}
\usepackage[german]{babel}
\usepackage[
tmargin=30mm,
bmargin=30mm,
lmargin=30mm,
rmargin=30mm
]{geometry}

\setcounter{secnumdepth}{-\maxdimen} % remove section numbering
\pagestyle{empty} % no page numbers

\begin{document}

\begin{center}
  {\large Studienleistung in Mikroökonometrie zum Thema} \\ \vspace{1em}
  {\LARGE $title$}
\end{center}

\section{Bearbeitungshinweise}

\begin{itemize}
  \item Ihre Bearbeitung ist in zwei Teile aufgeteilt, die Sie jeweils im Moodle über die dortigen Eingabemasken als einzelne \texttt{.pdf}-Datei hochladen. Abgaben per E-Mail sind nicht möglich.
  \begin{enumerate}
    \item Bis zum 20.12.2024 geben Sie Ihre Beschreibung der Daten ab (Teil 1 der Studienleistung).
    \item Bis zum 17.01.2025 geben Sie Ihre Formulierung der Methoden und Ihre Modellierung ab (Teil 2 der Studienleistung).
  \end{enumerate}
  \item Für den Erwerb der Studienleistung müssen alle Fragestellungen sinnvoll bearbeitet und ansprechend präsentiert in zwei Abgaben fristgerecht eingereicht werden.
  \item Es wird empfohlen, die Bearbeitung mit \texttt{R Markdown} zu erstellen. 
  \item Am 21.01.2025 gibt es im Rahmen der Praktischen Übung eine Abschlussdiskussion der Abgaben. Ihre Teilnahme daran ist eine Voraussetzung für den Erwerb der Studienleistung.
\end{itemize}

$body$

\end{document}